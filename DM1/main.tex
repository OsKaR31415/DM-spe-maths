\documentclass[10pt]{scrartcl}


\usepackage{inputenc}
\usepackage[T1]{fontenc}
\usepackage[french]{babel}

\usepackage{amsmath, amssymb, amsfonts}

\usepackage{enumitem}
\usepackage[top=2cm, bottom=2.5cm, right=1.5cm, left=1.5cm]{geometry}

\usepackage{xcolor}
\usepackage{fancyhdr}
\usepackage{listings}
\usepackage{euler}

\definecolor{codegreen}{rgb}{0,0.6,0}
\definecolor{codegray}{rgb}{0.5,0.5,0.5}
\definecolor{codepurple}{rgb}{0.58,0,0.82}
\definecolor{backcolour}{rgb}{0.95,0.95,0.92}

\lstdefinestyle{mystyle}{
    backgroundcolor=\color{white},   
    commentstyle=\color{red!90!black},
    keywordstyle=\color{orange!90!black},
    numberstyle=\scriptsize\ttfamily\color{gray},
    stringstyle=\color{green!70!black},
    basicstyle=\color{black},
    breakatwhitespace=true,
    breaklines=true,
    captionpos=b,
    keepspaces=true,
    numbers=left,
    numbersep=10pt,
    showspaces=false,
    showstringspaces=false,
    showtabs=true,
    tabsize=2
}

\lstset{style=mystyle}


\title{Devoir Maison de mahtématiques N$^{\circ{}}$1}
\author{Oscar Plaisant}

%%%%%%%%%%%%%%%%%%%%%%%%%%%%%%%%%%%%%%%%%%%%%%%%%%%%%%%%%%%%%%%%%%%%%%%%%%%%%%%%%%%%%%%%%%%%%%%%%%%%%%%%%%%%%%%%%%%%%%%%%%%%%%%%%%%%%%%%%%%%%%%%%%%%%%%%%%%%%%%%%%%%%%%%%%%%%%%%%%%%%%%%%%%%%%%%%%%%%%%%%%%%%%%%%%
\begin{document}

\pagestyle{fancy}

\renewcommand{\headrulewidth}{0pt}
\fancyhead[L]{}
\fancyhead[C]{}
\fancyhead[R]{}

\renewcommand{\footrulewidth}{1pt}
\fancyfoot[L]{D.M. N$^\circ{}$1}
\fancyfoot[C]{\thepage{}}
\fancyfoot[R]{Oscar Plaisant}

\thispagestyle{empty}
\maketitle{}


%%%%%%%%%%%%%%%%%%%%%%%%%%%%
%%%%%%%%%%%%%%%%%%%%%%%%%%%%
% \newpage{}%%%%%%%%%%%%%%%%
\section*{Exercice 1}%%%%%%%


\begin{enumerate}[label=\textbf{\arabic*{}.}]
    \item\subsection*{}

    \begin{enumerate}[label=\textbf{\alph*{})}]
        \item Déterminer $u_0$, $u_1$, $u_2$ et $u_3$.
        $$
            \begin{array}{l|l}
                u_0 = 200 & u_1=\dfrac{u_0}{2} + 180 = 190\\[3ex]
                \hline\\
                u_2 = \dfrac{u_1}{2} + 180 = 185 & u_3 = \dfrac{u_2}{2} + 180 = 182.5
            \end{array}
        $$

        \item Peut-on confirmer les espérences de Marc ?

        \begin{center}
            Il semble que non.
        \end{center}

        \item Donner la relation de récurrence liant $u_{n+1}$ à $u_n$.

        $$
            u_n : 
            \left\{
            \begin{array}
                {l}u_0 = 200 \\ u_{n+1} = \dfrac{u_n}{2} + 180
            \end{array}
            \right.
        $$
    \end{enumerate}

    \item On note $(v_n)$ la suite définie sur $\mathbb{N}$ par $v_n=u_n - 360$

    \begin{enumerate}[label=\textbf{\alph*)}]
        \item Montrer que $\forall n\in\mathbb{N}, v_{n+1} = \dfrac{1}{2}v_n$

        $$
            \begin{array}{rl}
                v_{n+1} = \dfrac{1}{2}v_n & \iff{} v_{n+1} = u_{n-1} - 360\\[3ex]
                 & \iff{} v_{n+1} = \dfrac{u_n}{2}+180-360\\[3ex]
                 & \iff{} \dfrac{u_n}{2} - 180\\[3ex]
                 & \iff{} \dfrac{u_n - 360}{2}\\[3ex]
                 & \iff{} \dfrac{v_n}{2}\\[3ex]
                v_{n+1} & \iff{} \dfrac{1}{2}v_n\\
            \end{array}
        $$
        

        \item En déduire une expression du terme général de $(v_n)$

        On sait que :

        $$
            v_{n+1} = \dfrac{1}{2}v_n
        $$

        Et que :

        $$
            \begin{array}{rl}
                v_0 &= u_0 - 360\\
                    &= 200 - 360\\
                    &= -160\\
            \end{array}
        $$

        On peut donc poser :

        $$
            v : \left\{
            \begin{array}{l}
                v_0 = -160\\
                v_{n+1} = \dfrac{1}{2}v_n
            \end{array}
            \right.
        $$

        Et on à donc :

        \[
            v_n = -160\left(\frac{1}{2}\right)^n \iff{} v_{n+1} = \left(\frac{1}{2}\right) (-160)\left(\frac{1}{2}\right)^n \iff{} v_{n+1} = \frac{1}{2} v_n
        \]

        On peut donc dire que :

        \[
            v_n = -160\left(\frac{1}{2}\right)^n
        \]

        \item En déduire une expression générale de $(u_n)$

        On sait que $u_n = v_n + 360$. On en déduit que :

        \[
            u_n = -160\left(\frac{1}{2}\right)^n + 360
        \]

        \item Préciser ce que peut espérer Marc en étudiant les variations et la limite de $(u_n)$.

        On sait que $u_n = -160\left(\frac{1}{2}\right)^n + 360$

        On peut dire que :

        \[
            -160\left(\frac{1}{2}\right)^n = -160\left(\frac{1}{2^n}\right)
        \]

        Or, la suite $n \mapsto{} 2^n$ est strictement croissante et est toujours positive. Donc, la suite $n \mapsto{} \frac{1}{2^n}$ est strictement décroissante (puisque la fonction inverse est strictement décroissante sur $\mathbb{R}^{+*}$). On peut aussi dire que $\displaystyle{}\lim_{n\rightarrow{} +\infty{}}\left(\dfrac{1}{2^n}\right) = 0$ car $\displaystyle{}\lim_{n\rightarrow{}+\infty{}}\left(2^n\right) = +\infty{}$ et que $\displaystyle{}\lim_{n\rightarrow{}+\infty{}}\left(\dfrac{1}{n}\right) = 0$. On peut donc dire que $\displaystyle{}\lim_{n\rightarrow{}+\infty{}}\left(-160\frac{1}{2^n}\right) = 0$, et finalement que : 

        \[
            \lim_{n\rightarrow{} +\infty{}}\left(-160\left(\frac{1}{2}\right)^n+360\right) = 360
        \]

        La suite $(u_n)$ tends donc vers $360$ en $-\infty{}$
    \end{enumerate}

    \item \begin{enumerate}[label=\textbf{\alph*{})}]
        \item  Plus généralement, en notant $C$ le montant initial des économies de Marc, exprimer $u_n$ en fonction de $n$ et $C$

        \[
            u_n = (C - 360)\left(\frac{1}{2}\right)^n + 360
        \]

        \item Le montant initial de ses économies influe-t-il sur la limite de $(u_n)$ ? sur ses variations ?

        En reprenant le raisonnement dévoloppé à la question 2.d), on peut montrer que $\displaystyle{}\lim_{n\rightarrow{}+\infty{}}\left(\frac{1}{2^n}\right) = 0$

        Or, on sait que quelque soit $\psi$ appartenant à $\mathbb{R}$ :

        \[
            \displaystyle{}lim_{n\rightarrow{}+\infty{}}\left(\psi\frac{1}{2^n}\right) = 0
        \]

        On en déduit donc que le montant initial des économies de Marc n'influe pas sur la limite de $(u_n)$.

        Le montant initial des économies de Marc n'influe pas sur les variations de $(u_n)$ si il est inférieur à 360. Si le montant initial $C$ dépasse 360, le sens de variation de $(u_n)$ est inversé (elle devient décroissante) mais sans changer de limite en $+\infty{}$. Si $C = 360$, la suite devient constante (puisque tous les termes sont annulés sauf le $+360$, ce qui fait que $C=360 \implies{} \forall n \in \mathbb{N}u_n = 360$)

    \end{enumerate}
\end{enumerate}







%%%%%%%%%%%%%%%%%%%%%%%%%%%%
%%%%%%%%%%%%%%%%%%%%%%%%%%%%
\newpage{}%%%%%%%%%%%%%%%%%%
\section*{Exercice 2:}%%%%%%



\begin{enumerate}[label=\textbf{\arabic*{}.}]
    \item Déterminer les valeurs de $u_3$ et $u_4$.
    
    Le nombre de lapins au troisième mois est 2 car le couple initial à engendré un nouveau couple.

    $$u_3 = 2$$

    Le nombre de lapins au quatrième mois est 3 car le premier couple à engendré un nouveau couple, tandis que le second couple n'est pas encore pret à engendrer un nouveau couple.

    $$u_4 = 3$$

    \item Expliquer pourquoi les mois suivants, la suite vérifie la relation de récurrence $u_{n+2} = u_{n+1} + u_n$

    Le nombre de lapins au mois $n$ est égal à la somme du nombre de lapins au mois $n-1$ et du nombre de lapins matures (c'est à dire aptes à faire des enfants) au mois $n$. Le nombre de lapins matures au mois $n$ est égal au nombre de lapins nés au mois $n-2$ ou avant, soit $u_{n-2}$.

    On peut donc dire que :

    \begin{tabular}{rcl}
        (lapins au mois $n$) & = & (lapins au mois $n-1$) + (lapins matures au mois $n-1$)\\
                            & = & (lapins au mois $n-1$) + (lapins nés avant le mois $n-1$)\\
                            & = & (lapins au mois $n-1$) + (lapins au mois $n-2$)\\[3ex]

    \end{tabular}

     
    En traduisant cela en langage mahtématique, on obtient: 
    $$u_n = u_{n-1} + u_{n-2} \iff u_{n+2} = u_{n+1} + u_n$$

    \item Le(s)quel(s) des algorithmes ci-dessous a (ont) permis d'obtenir les premier termes suivants de la suite, et pour quelle valeur de $n$?

    Le programme ci-dessous permet de déterminer quel algorithme donne le bon résultat. Pour trouver la valeur de $n$ correspondante, il suffit de tester quelques valeur jusqu'à ce que au moins l'un des algorithmes renvoient la bonne valeur
    
    \lstinputlisting[language=Python]{Test_Algos_1-2-3.py}

\end{enumerate}



\end{document}

