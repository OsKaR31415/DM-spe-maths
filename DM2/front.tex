%%%%%%%%%%%%%%%%%%%
% =============== %
% = Les paquets = %
% =============== %
%%%%%%%%%%%%%%%%%%%

\documentclass[14pt,aspectratio=169]{beamer} % settings généraux
\usepackage{etex} % pour ajouter de la mémoire au compilateur si le document est trop lourd




%%%%% symboles, langue %%%%%

\usepackage[utf8]{inputenc} % encodage utf8
\usepackage[french]{babel} % caractères francais (é,è,à,ê...)
\usepackage[T1]{fontenc} % mode francais
\usepackage{euler} % fonte du mode math
\usepackage{ulem} % symboles et polices
\usepackage{amsmath,amsthm,amsfonts,amssymb,amscd} % ams(
\usepackage{skull}
\usepackage{fontawesome} % symboles
\usepackage{tipa} % symboles

\usepackage{mathrsfs} % fontes de maths
\usepackage{enumitem} % plus de libereté dans "enumerate"
\usepackage{array}



%%%%% dessin, mise en page %%%%%

\usepackage{xcolor} % pour des couleurs évoluées
\usepackage{graphicx} % pour des dessins
\usepackage{hyperref} % pour des hyper-liens
% \usepackage{multicol} %


% Packages Figures et graphiques
% \usepackage{graphics} % inclusion de figures
\usepackage{graphicx} %inclusion de figures

\usepackage{tikz} % Tikz !
\usepackage{pixelart} % pour dessiner du pixel-art
\usepackage{pdfpages} % pour inclure des pages pdf

\definecolor{gold}{RGB}{255, 215, 0}

\definecolor{pixel 0}{HTML}{FFFFFF} % dead
\definecolor{pixel 1}{HTML}{000000} % alive
\definecolor{pixel 2}{HTML}{880000} % will die
\definecolor{pixel 3}{HTML}{6EA0F0} % will birth
\definecolor{pixel c}{HTML}{6B0000} % central cell
\definecolor{pixel n}{HTML}{FFA0A0} % neighbours

\newcommand{\pixelart}[2]{%
        \begin{tikzpicture}[scale=#2]
        \foreach \line [count=\y] in #1 {
            \foreach \pix [count=\x] in \line {
            \draw[fill=pixel \pix] (\x,-\y) rectangle +(1,1);
            }
        }
        \end{tikzpicture}
    }

\newcommand{\framePixelScale}{.5}

\newcommand{\gridFrame}[2]{%
\begin{frame}{#1}
    \begin{center}
        \vspace*{-1cm}\hspace*{-1.2cm}
        \pixelart{#2}{\framePixelScale}
    \end{center}
\end{frame}
}


\newcommand{\DIE}{{\color{pixel 2}$\skull$}}
\newcommand{\STAY}{{\color{black}\faThumbsOUp}}
\newcommand{\BIRTH}{{\large\color{pixel 3}$\bigstar$}}



\setbeamercolor{block body example}{bg=red!20!white}
\setbeamercolor{block title example}{fg=red, bg=red!40!white}

\newcommand{\inputanimation}[1]{\input{_sections/animations/#1}}

% \renewcommand{\inputanimation}[1]{#1}

%% les fontes
%\DeclareTextFontCommand{\angles}{\fontfamily{bch}\selectfont}
%\DeclareTextFontCommand{\arrondi}{\fontfamily{ppl}\selectfont}
%\DeclareTextFontCommand{\fin}{\fontfamily{pag}\selectfont}
%\DeclareTextFontCommand{\gras}{\fontfamily{bch}\selectfont}
%\DeclareTextFontCommand{\anglesfin}{\fontfamily{pnc}\selectfont}

\setitemize[0]{font=\bfseries, label=\bwpixelart[scale=.04, color=UBCblue, raise=-.3ex]{%
        % 11111111011
        % 11111111011
        % 00000000011
        % 00000000011
        % 11000000011
        % 11000000011
        % 11000000011
        % 11
        % 11
        % 11001111111
        % 11001111111
        00000111100
        01101000010
        10001000010
        10000110000
        10010000110
        10010001001
        01100001001
        00001100001
        01000010001
        01000010110
        00111100000
    }
}

